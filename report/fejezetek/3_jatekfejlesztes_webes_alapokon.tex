\begin{MyChapter}{Játékfejlesztés webes alapokon}	
	% TODO - leírni mi lesz a fejezetben
	% TODO - először szót kell ejteni néhány témáról: html5, stb: melyek a webes környezet miatt kapcsolódnak a játékfejlesztéshez is.

	\begin{MySection}{Webes fejlesztés}
		% TODO 	- kialakulása, néhánya általános dolog
		
		Az internet fejlődésével a böngészőkben játszható játékok nem csak helyet kaptak, de szinte a legnagyobb közönséghez jutottak el. A hozzáférésük sokkal könnyebb, mint a boltokban megvásárolt játékoknak, hiszen szimplán csak egy webcímre szükséges ellátogatni, majd magában a böngészőben zajlik a játékmenet. Ezen felül a szociális hálózatok elterjedése szintén felgyorsította a böngészős játékok egyre több emberhez való eljutását.
		
		Azonban a böngészők erőforrásai kifejezetten korlátozottak voltak, komplexebb alkalmazásokhoz már nem biztosítottak elég funkciót. Ezért a fejlesztők különböző cégek által készített plug-in-eket, programokat használtak, mint például a Microsoft Silverlight, vagy az Adobe Flash. Ezek a kiegészítések már komolyabb grafikus tartalom megjelenítésére voltak képesek, viszont az összes használónak szükséges telepítenie a megfelelő plug-in-t.  
		
		% használt linkek:
		% https://en.wikipedia.org/wiki/Browser_game
		% https://www.awwwards.com/current-state-and-the-future-of-html5-games.html

	\end{MySection}

	\begin{MySection}{HTML5}
		% TODO 	- mi ez, miért jobb mint ami eddig volt
		
		A HTML5 fő célja a fentebb említett funkcióknak az egységesítése volt, anélkül, hogy a felhasználóknak különböző plug-in-eket kellene használniuk ahhoz, hogy megfelelően működjenek az egyes elemek a böngészőben. A funkciók szabványosítása folyamatosan zajlik, így bár már most is megfelelő platformot nyújt a webes játékok, alkalmazások számára, a jövőben még ennél is népszerűbbé válhat.
		
	\end{MySection}

	\begin{MySection}{Javascript}
		% TODO 	- mi ez, miért jó
		% TODO 	- kialakulása
	\end{MySection}

	\begin{MySection}{Typescript}
		% TODO 	- mi ez, miért jó
		% TODO 	- kialakulása
	\end{MySection}

	\begin{MySection}{Általános célú javascript keretrendszerek}
		% TODO 	- nem játékfejlesztéshez
		% TODO 	- angular, vue.js, react, egyéb
	\end{MySection}

	\begin{MySection}{Webes játékfejlesztés}
		% TODO 	- keretrendszerrel vagy nélküle
		% TODO 	- szükséges egyéb dolgok pl.: grafikai pakkok (al-al-fejezetek), hangok, texturák, objektumok
		% TODO 	- nem kell túl sokat ide írni, csak úgy általánosságban erről a dologról, mi kellhet hozzá
	\end{MySection}

	\begin{MySection}{Webes játékfejlesztés keretrendszerek nélkül}
		% TODO 	- "pure js"-ben
		% TODO 	- webgl? (lehet h ez is keretrendszer)
	\end{MySection}

	\begin{MySection}{Webes játékfejlesztés keretrendszerekkel}
		% TODO 	- keretrendszerek miért jók
		% TODO 	(- összehasonlítás menete(?))
		% TODO 	- felsorolni (tovább bontani al-al-fejezetekre ha megoldható)
		% TODO 	- megemlíteni melyik miért jó, miben más, használatuk, esetleg kipróbálás, személyes tapasztalatok
		% TODO 	- összehasonlítás eredménye / összegzése az előzőnek
		% TODO 	- dönteni melyiket használom
	\end{MySection}
	
\end{MyChapter}