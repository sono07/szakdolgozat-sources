\begin{MyChapter}{Bevezetés}
	% bevezetés megír
	% elég a legvégén, leírni miért ezt a témát választottad, miért fontos/jó ez, dolgozat szerkezetének vázolása (miről lesz szó)
	
	A szakdolgozatom témája a modern számítógépes játékfejlesztéssel kapcsolatos irodalomkutatás, valamint egy Tower Defense játék megvalósítása HTML5 alapokon.
	
	Ennek fényében kitérek a játékfejlesztés történelmére, valamint a technológia fejlődésére, amely gyorsasága egyre inkább érzékelhető napjainkban. Vélhetően nagy befolyással volt erre a fejlődésre a szórakoztatóipar jelentős részét kitevő videojátékok iparága is. 
	
	Játékfejlesztés esetén gyakorta felvetődik a kérdés, hogy szükséges-e saját motort alkotnunk, vagy a napjainkban már egyre népszerűbbnek számító előre elkészített, harmadik féltől származó játékmotorokat használjunk. A szakdolgozatomban szó lesz általánosságban a játékfejlesztéshez használható technológiákról, és mind a keretrendszerekkel, mind az anélküli játékfejlesztésről.
	
	Webes tekintetben részletesebben olvashatunk a játékok készítéséhez gyakran használt programnyelvekről, a játékmotor nélküli fejlesztésről, ezek előnyeiről és hátrányairól, illeve néhány konkrét keretrendszer bemutatásáról. A saját motor készítése bár sok szempontból kimondottan előnyös, a megvalósítás folyamata akár több évig is tarthat, mielőtt a játékot egyáltalán elkezdenénk fejleszteni, így kezdőként én egy már meglévő játékmotor használata mellett döntöttem. A lehetőségek taglalása után tehát kiválasztok egy keretrendszert, amelyben a játékot elkészítem. Ehhez a következő négy játékmotort hasonlítom össze, mely a döntést elősegíti: \textit{ImpactJS Engine}, \textit{Modd.io}, \textit{Construct 2}, \textit{Phaser}.
	
	A szakdolgozatban részletesen végigvezetem az olvasót az általam készített játék fejlesztésének lépésein, a nehézségeken, melyeket meg kellett oldanom.

	Ezenkívül az elkészült játék bemutatása után az esetleges továbbfejlesztési lehetőségeket is taglalni fogom, valamint véleményemet a választott motorról, beleértve azt is, hogy mennyire tartom alkalmasnak kisebb-nagyobb projektekhez, ajánlanám-e a használatát általánosságban.
	
\end{MyChapter}