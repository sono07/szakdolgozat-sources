\begin{MyChapter}{Összegzés}
	
	% TODO - elég a legvégén, szerzett tapasztalatok, kihangsúlyozni az elért eredményeket (főleg a sajátokat!), alkalmazás továbbfejlesztési lehetőségei
	
	% TODO Hasonló szerepe van, mint a bevezetésnek. Itt már múltidőben lehet beszélni.	A szerző saját meglátása szerint kell összegezni és értékelni a dolgozat fontosabb eredményeit. Meg lehet benne említeni, hogy mi az ami jobban, mi az ami kevésbé jobban sikerült a tervezettnél. El lehet benne mondani, hogy milyen további tervek, fejlesztési lehetőségek vannak még a témával kapcsolatban.
	
	% TODO összegzésben Phaser 3-mal kapcsolatos tapasztalatokat(általánosan h erre a célra mennyire felel meg, ajánlanám-e ezt ilyen jellegű program készítésére)
	
	% TODO Továbbfejlesztése lehetoseg:	Ui-ra információs dialóg amikor a torony stb felé visszük az egeret
	
	% TODO ### Továbbfejlesztés ###
	% - hangok kimaradtak
	% - teljes tesztlefedettsége, pl az objectekre is
	% - a konfigurálthaót értékeket egy külső fájlból betölteni így manuálisan akár a felhasználó is módosíthatja az értékeket saját kénye kedve szerint.
	% - effektekhez vizuális részt is adni, pl a slow esetén megfagyvottként látszódjon a enemy.
	
	% TODO továbbfejlesztés:
	% - Finomhangolási értékeken lehetne még módosítani, mert tapasztalat alapján elég könnyűre sikerült, de ez akár egy különálló kutatás tárgyát is képezhetné, hiszen a tökéletes egyensúly megtalálsa egy optimalizálási feladatként is felfogható.
	% - Ha több féle, különböző képességű enemy lenne, akkor nehezebb lenne / újra kellene finomhangolni.
\end{MyChapter}