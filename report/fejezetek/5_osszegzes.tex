\begin{MyChapter}{Összegzés}
	
	% elég a legvégén, szerzett tapasztalatok, kihangsúlyozni az elért eredményeket (főleg a sajátokat!), alkalmazás továbbfejlesztési lehetőségei
	
	% A szakdolgozat során elsődleges céljaim közé tartozott a keretrendszerrel való megismerkedés.
	A \myref{Célok meghatározása} fejezetben leírt játékkal kapcsolatos terveimet nagyrészt sikerült megvalósítanom. A játékfejlesztés folyamata egyáltalán nem volt problémamentes, azonban a végeredménnyel összességében elégedett vagyok, elvégre ez még csak az első játék volt, amit valaha készítettem.
	Abban pedig egészen biztos vagyok, hogy a szakdolgozat elkészítése során szerzett rengeteg hasznos ismeretet alkalmazni tudom majd a jövőben is, illetve minden bizonnyal továbbfejleszteni is.
	
	% összegzésben Phaser 3-mal kapcsolatos tapasztalatokat(általánosan h erre a célra mennyire felel meg, ajánlanám-e ezt ilyen jellegű program készítésére)
	\begin{MySection}{Phaser 3-mal kapcsolatos tapasztalatok}
		Sok szempontból szimpatikus motor, könnyű a használata, tanulhatósága, viszont szerintem kisebb projektekhez alkalmas inkább, nagyobb, komplexebb projektek megvalósítására nem ajánlanám.
	\end{MySection}
	
	\begin{MySection}{Továbbfejlesztési lehetőségek}
		Bár jelenleg működik a játék, tartalmazva majdnem minden előzetesen meghatározott célkitűzést, viszont még rengeteg fejlesztési lehetőség rejlik benne, változatosabbá lehetne tenni a játékmenetet.
		A fejlesztés során, illetve a manuális tesztelés közben, azaz amikor játszottam vele, több ötletem is támadt a továbbfejlesztéssel kapcsolatban:
		
		\begin{itemize}
			\item Elsősorban, amikor először nagyvonalakban késznek nyilvánítottam a játékot, azután tűnt fel, hogy egyáltalán nem is gondoltam a hangokra, azok pótlása segíthet egy jobb játékbeli hangulatot megteremteni a játékosok számára.
			
			\item Vizuális effektek hozzáadása szintén hozzájárulhat a pozitívabb felhasználói élményhez. Például slow effekt használata esetén megfagyva látszódna az ellenség.
			
			\item Játékosok szemszögéből hasznos lehetne még mondjuk egy információs ablak, ami leírást adhat játékban résztvevő tornyokról és egyéb elemekről, mert jelenleg ezek nincsenek részletezve.
			
			\item Többféle, esetleg különböző képességekkel rendelkező ellenség bevezetése javíthatna a változatosságon.
			
			\item Időhiányában nem volt idő mindent letesztelni, így idesorolnám a teljes tesztlefedettség elérését is.
			
			\item A finomhangolási értékeken lehetne még módosítani, mert tapasztalat alapján elég könnyűre sikerült, de ez akár egy különálló kutatás tárgyát is képezhetné, hiszen a tökéletes egyensúly megtalálása egy optimalizálási feladatként is felfogható.
			
			\item A konfigurálható értékeket egy külső fájlból betölthetővé lehetne tenni, így manuálisan akár a felhasználó is módosíthatná az értékeket saját kedve szerint, hogy mennyire szeretne nehéz, vagy épp könnyű játékmenetet.
			
			\item Mobilon is működőképes a játék, de az ``egér'' kezeléssel problémák vannak, viszont ez javítható lehet. Elsősorban számítógépen használhatónak terveztem a játékot, de további elérendő cél lehetne mobilalkalmazásként is kiadni.
		\end{itemize}
		
	\end{MySection}

\end{MyChapter}