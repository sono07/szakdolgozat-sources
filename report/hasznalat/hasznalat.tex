% !TEX encoding = UTF-8 Unicode
\newpage

\section*{CD-melléklet tartalma}
% használati útmutató megír

A szakdolgozat \LaTeX\ segítségével készült. A dolgozat forrásfájljai a \texttt{report/sources} mappában találhatóak.
A dolgozat PDF-re fordított változata pedig a \texttt{report/pdf} mappában található meg \texttt{report.pdf} néven.

A dolgozat során elkészült játék forrásfájljai a \texttt{game/sources} mappába kerültek elhelyezésre.
A program fordításához a \textit{Node.js}-re, \textit{NPM csomagkezelő}-re, illetve internetkapcsolatra van szükség.

A \texttt{game/sources} mappában egy konzolt szükséges először nyitni, itt az \texttt{npm install} parancsot kell futtatni, a szükséges függőségek telepítéséhez (ehhez internetkapcsolat szükséges).
Ezután ha a teszteket szeretnénk futtatni, akkor azt az \texttt{npm test} parancs segítségével tehetjük meg.
Az alkalmazás fejlesztői módban való elindításához az \texttt{npm start} parancsot kell futtatnunk, ami a fordítás után egy webszervert fog elindítani a gépünkön a $8080$-as porton. Ezután böngészőnkben a \url{localhost:8080} -as oldalra navigálva elérhetjük az alkalmazást.
A program optimalizáltan és kiadható formában fordított változatát az \texttt{npm run build} parancs futtatásával lehet elkészíteni. Ez az aktuális mappán belüli \texttt{dist} mappába fogja elhelyezni a fordított fájlokat. Ezt követően a kész fájlokat egy webszerverre másolva lehet használni. A kipróbálhatóság érdekében az \texttt{npm run serve-build} paranccsal tudunk futtatni ideiglenesen egy minimális funkciókkal bíró webszervert. Ez a parancs elméletileg az alapértelmezett böngészőnkben meg is nyitja az alkalmazást mely a \url{http://127.0.0.1:4949/dist/} címen érhető el.

Az előre lefordított alkalmazás a \texttt{game/dist} mappában található, ennek használatához előzőekben említetthez hasonlóan szükségünk van egy webszerverre.

A játék használatáról részletesebben a dolgozatban a \myref{Használat} fejezetben található útmutatás.
